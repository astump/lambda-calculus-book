\chapter{Simply Typed Lambda Calculus}

\section{Syntax for types}
\label{sec:synstlc}

We assume a non-empty set $B$ of base types.  These are just any
mathematical objects we wish, that will play the role of atomic
(indivisible) types.  We will use $b$ as a meta-variable for elements
of type $B$.  Similarly as for our metavariables for
$\lambda$-calculus variables (see the start of
Section~\ref{sec:synlam}), we will adopt the convention that different
meta-variables refer to different base types, in any particular
meta-linguistic discussion. \index{base types} The syntax of types is
then:
\[
\textit{simple types}\ T\ ::= \ b\ |\ T \to T'
\]
\noindent There is one parsing convention for simple types, which is
that arrow is right-associative.  So a type like $a \to b \to c$
should be parsed as $a \to (b \to c)$.

Let us consider some examples of simple types.  We might have the type
$\textit{bool} \to \textit{bool}$ for boolean negation, and other
unary (1-argument) boolean operations.  Similarly, a type like
$\textit{bool} \to \textit{bool} \to \textit{bool}$ could describe
conjunction, disjunction, and any other binary boolean operations.
For a higher-order example, a type like $(\textit{nat} \to
\textit{bool}) \to \textit{nat}$ could be the type for a minimization
function \textit{minimize}, where $\textit{minimize}\ p$ returns the
smallest natural number $n$ such that $p\ n$ returns \textit{true}.

Now, it will happen that our notion of typing will not allow
interesting computations with values of atomic types like
\textit{bool}.  So we will not actually be able to type functions like
the ones just described in pure simply typed lambda calculus (STLC).
But STLC is the right framework for characterizing the functional
behavior (via arrow types $T \to T'$) of lambda terms, and thus forms
the core of most other more advanced type systems, including ones
where types like \textit{bool} are definable within the system.

\section{Realizability semantics of types}
\label{sec:stlcrealizability}

One very natural way to understand a type is as a specification
of the behavior of programs.  For example, in a programming
language like Java, suppose a function is declared with the signature

\begin{verbatim}
  int f(int x, int y);
\end{verbatim}

\noindent Then intuitively, the meaning of this is that function
\verb|f| expects two integers \verb|x| and \verb|y| as input and, if
it terminates normally (without raising an exception, diverging,
etc.), then it will return an integer as output.

This idea that a type is a form of specification for programs can be
made precise for STLC using a form of \emph{realizability semantics}\index{realizability}.  This semantics
was introduced by Kleene for intuitionistic arithmetic~\cite{kleene45}.  To explain
this further, we first need this notion:

\begin{definition}[$\beta$-expansion closed]
  A set $S$ of closed terms is $\beta$-expansion closed if $t\in S$ and $t' \leadsto t$ imply $t' \in S$, for closed $t'$. \index{$\beta$-expansion closed}
\end{definition}

\noindent Such a set $S$ is closed under $\beta$-expansion in the
sense that one cannot leave $S$ by following $\beta$-expansion steps
to closed terms.

Figure~\ref{fig:stlcrealize} defines an interpretation $\interp{T}$
for any simple type $T$, assuming a function $I$ which interprets the
base types of $B$.  We require that $I(b)$ is a $\beta$-expansion
closed set, for every $b\in B$.


The values computed by the semantic function and $I$ are sets of closed terms.  So mathematically,
writing \textit{Types} for the set of all simple types and \textit{ClosedTerms} for the set
of all closed terms of untyped $\lambda$-calculus (and using the standard notation $\mathcal{P} S$ for
the set of all subsets of a set $S$), we have:
\begin{itemize}
\item $\interp{-} \in \textit{Types} \to \mathcal{P}\ \textit{ClosedTerms}$
\item $I \in B \to \mathcal{P}\ \textit{ClosedTerms}$
\end{itemize}

\noindent Given that $I(b)$ is required to be $\beta$-expansion closed, we have:

\begin{lemma}
\label{lem:betaexpclosed}
  $\interp{T}$ is $\beta$-expansion closed,
  for all $T$.
\end{lemma}
\begin{proof}
  The proof is by induction on $T$.  If $T$ is a base type $b$, then
  $\interp{T} = I(b)$, which is $\beta$-expansion closed by
  assumption.  So assume $T$ is an arrow type $T_1 \to T_2$, and
  assume $t\in\interp{T}$, and closed $t'\betaa^* t$.  We must show
  $t'\in\interp{T}$.  For that, assume $t''\in\interp{T_1}$, and show
  $t'\ t''\in\interp{T_2}$.  By the induction hypothesis, $\interp{T_2}$
  is $\beta$-expansion closed.  So to show $t'\ t''\in\interp{T_2}$,
  it suffices to show that $t\ t''\in\interp{T_2}$, since $t'\ t''$
  reduces to $t\ t''$ (since $t'\betaa^* t$).  But $t\ t''\in\interp{T_2}$
  because $t\in\interp{T_1\to T_2}$ by assumption (and $t''\in\interp{T_1}$).
  \end{proof}

\begin{figure}
\[
\begin{array}{lll}
\interp{b} & = & I(b) \\
\interp{T_1\to T_2} & = & \{ t\in\textit{ClosedTerms}\ |\ \all{t' \in \interp{T_1}}{(t\ t') \in\interp{T_2}} \}
\end{array}
\]
\caption{Realizability semantics of types, with respect to an assignment $I$ of meanings for base types}
\label{fig:stlcrealize}
\end{figure}

Recall that the notation $t \downarrow$
(Definition~\ref{def:normalizing}) means that term $t$ is normalizing:
there exists some $t'$ such that $t\betaa^* t'$ and $t'$ is in normal
form (i.e., does not reduce to any term).

\begin{definition}
\label{def:normterms}
  Define $\mathcal{N}$ to be $\{t \in \textit{ClosedTerms}\ |\ t \downarrow\}$.
\end{definition}

\begin{lemma}
\label{lem:norminterp}
  If $I(b) = \mathcal{N}$ for all $b\in B$, then $\interp{T}$ is non-empty and a subset of $\mathcal{N}$ for
  all $T$.
\end{lemma}
\begin{proof}
  The proof is by induction on $T$.  For the base case, suppose $T$ is
  some $b\in B$.  Then $\interp{T} = \interp{b} = I(b) = \mathcal{N}$.
  And this set contains an element, for example $\lam{x}{x}$.  For the
  step case, suppose $T$ is a function type $T_1 \to T_2$, and assume
  $t\in\interp{T}$.  We must show $t\downarrow$.  Since $\interp{T_1}$ is
  non-empty by induction hypothesis, we may select some $t'\in\interp{T_1}$.
  Then by the semantics of arrow types, $t\ t'\in\interp{T_2}$.  By the induction
  hypothesis (for $T_2$), this implies that $t\ t'\downarrow$.  In turn, this
  implies $t\downarrow$ (proof omitted). % FIXME
  For an element of $\interp{T_1\to T_2}$, there exists $t\in\interp{T_2}$ by
  the induction hypothesis.  We then have $\lam{x}{t}\in\interp{T_1\to T_2}$,
  by the semantics of arrow types. 
\end{proof}   

\subsection{Examples}

Let us see some examples of the semantics.  Suppose that $B$ consists of two base types, $b$ and $b'$.  
Let us write $\mathbb{B}$ for the set of
Church-encoded booleans (Section~\ref{sec:churchenc}), and $\mathbb{N}$ for the set of
Church-encoded natural numbers.  Then define $I$ by:
\[
\begin{array}{lll}
  I(b) & := & \{ t\in\textit{ClosedTerms}\ |\ \exists\ b\in\mathbb{B}.\ t \leadsto^* b\} \\
  I(b') & := & \{ t\in\textit{ClosedTerms}\ |\ \exists\ n\in\mathbb{N}.\ t \leadsto^* n\} 
\end{array}
\]
\noindent

\textbf{Example.} First, we can observe that $\textit{not}\in\interp{b\to b}$.  To show this, it suffices to
  assume an arbitrary closed $t'$ with $t'\leadsto^*\mathbb{B}$, and show
  that $\textit{not}\ t'\leadsto^*\mathbb{B}$. Suppose $t'\leadsto^*\textit{true}$.
  Then we have
  \[
  \textit{not}\ t' \ \leadsto^*\  \textit{not}\ \textit{true}\ \leadsto^*\ \textit{false}
  \]
  \noindent And similarly, if closed $t'\leadsto^*\textit{false}$, we have
  \[
  \textit{not}\ t' \ \leadsto^*\  \textit{not}\ \textit{false}\ \leadsto^*\ \textit{true}
  \]

\textbf{Example.} Next, let us define this function:
  \[
  \textit{even}\ :=\ \lam{x}{x\ \textit{not}\ \textit{true}}
  \]
  \noindent Given a Church-encoded natural number $n$, this function iterates boolean negation
  $n$ times starting with \textit{true}.  This will result in \textit{true} iff $n$ is indeed even.
  Let us prove that $\textit{even} \in \interp{b' \to b}$.  Assume $t\in\interp{b'}$ and show $\textit{even}\ t \in \interp{b}$.
  Since $t\in\interp{b'}$, there is some natural number $n$ such that
  \[
  t \betaa^* \lam{f}{\lam{x}{\underbrace{f\ \cdots\ (f}_n\ x)}}
  \]
  \noindent Then we have
  \[
  \textit{even}\ t\ \betaeq\ t\ \textit{not}\ \textit{true}\ \betaeq\ \underbrace{\textit{not}\ \cdots\ (\textit{not}}_n\ \textit{true})
  \]
  \noindent We may easily prove that the latter term is $\beta$-equivalent to \textit{true} if $n$ is even, and to \textit{false} if $n$ is odd.



\textbf{Example.} Suppose there is a base type $b\in B$, and define $I(b)$ to be $\mathcal{N}$.
Then  $\lam{x}{\lam{y}{x}} \in \interp{b \to b}$.  To prove this, it suffices by the semantics
  of arrow types to assume $t'\in \interp{b}$, and
  show $(\lam{x}{\lam{y}{x}})\ t' \in \interp{b}$.  Since $\interp{b} = I(b)$,
  we are assuming closed $t' \downarrow$, and need to show $(\lam{x}{\lam{y}{x}})\ t'\downarrow$.
  For the latter:
  \[
  \underline{(\lam{x}{\lam{y}{x}})\ t'} \leadsto \lam{y}{t'}
  \]
  \noindent and the latter is normalizing (and closed) since $t'$ is. Also, $\lam{x}{\lam{y}{x}} \in \interp{b}$, since $\lam{x}{\lam{y}{x}}$ is in normal
  form (and hence normalizing), and closed. With this example, the same term is in $\interp{b}$ and $\interp{b \to b}$.
  Let us next consider an example (with the same interpretation $I$ for the base type $b$) where we have a term in $\interp{b}$ that is not in $\interp{b \to b}$.

  \textbf{Example.} We have
 $\lam{x}{x\ x} \in \interp{b}$, since $\lam{x}{x\ x}$ is in normal form and closed.
 But $\lam{x}{x\ x} \not\in \interp{b \to b}$.  To prove that, we must exhibit $t \in \interp{b}$ such
  that $(\lam{x}{x\ x})\ t$ is not in $\interp{b}$.  We may use $\lam{x}{x\ x}$ for that $t$, because
  we just observed that it is in $\interp{b}$, but $(\lam{x}{x\ x})\ (\lam{x}{x\ x})$ (i.e., the term $\Omega$) is definitely
  not in $\interp{b}$, since $\Omega$ is not normalizing.


\textbf{Example.} Finally, let us change the interpretation $I(b)$ to be $\{ t \in\textit{ClosedTerms}\ |\ t \betaa^* \lam{x}{x}\}$.  Then we
have an example opposite to the one we just found: a term in $\interp{b\to b}$ that is not in $\interp{b}$.
The term is again $\lam{x}{x\ x}$.  This term does not reduce to $\lam{x}{x}$, and so it is not in $\interp{b}$.
But it is in $\interp{b \to b}$.  To show that, assume $t\in\interp{b}$, and show $(\lam{x}{x\ x})\ t\in\interp{b}$.
Since $t\in\interp{b}$, we have $t\betaa^*\lam{x}{x}$.  Then we have the following reduction confirming
that the starting term is in $\interp{b}$:
\[
(\lam{x}{x\ x})\ \underline{t} \leadsto^* \underline{(\lam{x}{x\ x})\ \lam{x}{x}} \leadsto \underline{(\lam{x}{x})\ \lam{x}{x}} \leadsto \lam{x}{x}
\]


\section{Type assignment rules}

To obtain a computable approximation of the realizability semantics of the previous section,
we use a system of rules for deriving facts of the form $\Gamma \vdash t : T$; such facts
are called \emph{typing judgments}\index{typing judgment}.  Here, $\Gamma$
is a \emph{typing context}, with the following syntax:
\[
\textit{typing contexts}\ \Gamma\ ::=\ \cdot\ |\ \Gamma , x : T
\]
\noindent There is an empty context $\cdot$, and a context may be
extended on the right with a binding $x : T$.  This represents an
assumption that $x$ has type $T$.  We will type open terms (terms with
free variable occurrences) by making assumptions, in typing contexts,
about the types of their free variables.  The typing rules are in Figure~\ref{fig:stlctpassign}.

\begin{figure}
\[
\begin{array}{lll}
\infer{\Gamma\vdash x : T}{\textit{Find}\ x:T\ \textit{in}\ \Gamma} 

&

\infer{\Gamma\vdash \lam{x}{t} : T' \to T}
      {\Gamma,x:T'\vdash t:T}

&

\infer{\Gamma\vdash t_1\ t_2 : T}
      {\Gamma\vdash t_1 : T' \to T &
       \Gamma\vdash t_2 : T'}
      \\
      \\

\infer{\textit{Find}\ x:T\ \textit{in}\ (\Gamma, x:T)}{\ }

&

\infer{\textit{Find}\ x:T\ \textit{in}\ (\Gamma, y:T')}{\textit{Find}\ x:T\ \textit{in}\ \Gamma}

&

\ 
\end{array}
\]
\caption{Type-assignment rules for simply typed lambda calculus, with rules for looking up a variable declaration in the context $\Gamma$}
\label{fig:stlctpassign}
\end{figure}

\subsection{Examples}

An example typing derivation is given in Figure~\ref{fig:stlcex}.  Let us adopt the convention that
we do not show derivations of \textit{Find} judgments.  Thus, we will allow derivations to terminate
in applications of the variable rule (first rule in Figure~\ref{fig:stlctpassign}) with premise elided, as long
as that elided premise is actually derivable.

\begin{figure}
  \[
  \infer{\cdot \vdash \lam{x}{\lam{y}{x\ y\ y}} : (b \to b \to b) \to b \to b}
        {\infer{\cdot,x:b \to b \to b \vdash \lam{y}{x\ y\ y} : b \to b}
          {\infer{\Gamma \vdash x\ y\ y : b}
            {\infer{\Gamma \vdash x\ y : b \to b}
              {\infer{\Gamma \vdash x : b \to b \to b}{\ }
              & \infer{\Gamma \vdash y : b}{\ }}
            &\infer{\Gamma \vdash y : b}{\ }}}}
  \]
\caption{Example typing derivation in STLC, where $\Gamma$ abbreviates the typing context $\cdot,x:b \to b \to b, y : b$}
\label{fig:stlcex}
\end{figure}

\subsection{Soundness with respect to the realizability semantics}

We have given two meanings for types, and it is now time to relate
them.  In this section, we will prove soundness of the typing rules
(Figure~\ref{fig:stlctpassign}) with respect to the realizability
semantics (Figure~\ref{fig:stlcrealize}).

In logic generally, suppose we have two ways of defining the meaning
of some formulas, via sets $S_1$ and $S_2$ of formulas that are
considered (by the two semantics respectively) to be true.  Then $S_1$
is sound with respect to $S_2$ iff $S_1 \subseteq S_2$, and $S_1$ is
complete with respect to $S_2$ iff $S_2 \subseteq
S_1$.\index{sound}\index{complete} One way to think about this is as
if $S_1$ consists of statements made by a person, and $S_2$ consists
of statements that are true in reality.  Then soundness means that the
statements the person makes are, in fact, true.  So $S_1$ (the set of
statements the person makes) is a subset of $S_2$ (the statements that
are really true).  Completeness is the inverse of this: if a statement
is really true (i.e., in $S_2$), then the person affirms it (i.e., it
is in $S_1$).  It is easy to be sound: one affirms nothing.  It is
also easy to be complete: one affirms everything.  The ideal, which is
more difficult or even impossible to achieve, depending on the logic,
is to be both sound and complete.

To consider these properties for STLC, we need to formulate the
formulas that our two semantics are affirming.  The basic formula for
typing is $\vdash t : T$.  The corresponding formula for our
realizability semantics is $t \in \interp{T}$.  But the typing rules
make use of typing contexts $\Gamma$ and use more general formulas
$\Gamma \vdash t : T$.  So we will need a corresponding generalization
of the formulas of our realizability semantics.

\begin{definition}
  \label{def:substmodels}
  Define $\interp{\Gamma}$ to be the set of substitutions $\sigma$ such that
  whenever $\textit{Find}\ x:T\ \textit{in}\ \Gamma$ is derivable (with the rules of Figure~\ref{fig:stlctpassign}),
  then $\sigma(x) \in \interp{T}$.
\end{definition}

So $\sigma\in\interp{\Gamma}$ means that the substitution $\sigma$
maps variables to terms that are in the interpretations of the types
that $\Gamma$ says they have.  Let $\sigma\ t$ denote the 
application of the substitution $\sigma$ to $t$, with definition
given in Figure~\ref{fig:substsigma}.  Since we are assuming that the
range of $\sigma$ consists of closed terms (since $\interp{T}$ is a set
of closed terms for every $T$), we do not need to worry
about variable capture: none is possible.

\begin{theorem}
  \label{thm:stlcsnd}
  If $\Gamma\vdash t : T$, then for every $\sigma\in\interp{\Gamma}$, $\sigma\, t \in \interp{T}$,
  for all interpretations $I$ where $I(b)$ is $\beta$-expansion closed.
\end{theorem}
\begin{proof}
  The proof is by induction on the typing derivation.

  \case{ }
\[  \infer{\Gamma\vdash x : T}{\textit{Find}\ x:T\ \textit{in}\ \Gamma} 
\]
Assume $\sigma \in \interp{\Gamma}$.  By definition of the interpretation
of $\Gamma$, this means that $\sigma(x) \in\interp{T}$, and hence $\sigma\,x$, which equals $\sigma(x)$,
is in $\interp{T}$ as required.

  \case{ }
\[  \infer{\Gamma\vdash \lam{x}{t} : T' \to T}
      {\Gamma,x:T'\vdash t:T}
\]
Assume $\sigma\in\interp{\Gamma}$.  To show $\sigma\,\lam{x}{t}\in\interp{T'\to T}$, it suffices
by the definition of substitution and the semantics of arrow types to assume $t'\in\interp{T'}$,
and show that $(\lam{x}{\sigma\, t})\ t'$ is in $\interp{T}$.  Since $\interp{T}$ is $\beta$-expansion closed
by Lemma~\ref{lem:betaexpclosed}, it suffices to show $[t'/x]\sigma\, t\in\interp{T}$.

Let $\sigma'$ be the same as $\sigma$ except that $x$ is mapped to $t'$.
Then $\sigma'\in\interp{\Gamma,x:T'}$, since $\sigma'(x)\in\interp{T}$.  The desired conclusion
now follows directly by the induction hypothesis.
\end{proof}

\begin{figure}
  \[
  \begin{array}{lll}
    \sigma\, x & = & \left\{\begin{array}{ll}
                           \sigma(x),\textnormal{ if }x\in\textit{dom}(\sigma) \\
                           x,\textnormal{ otherwise}
                           \end{array} \right. \\
    \sigma\, (t_1\ t_2) & = & (\sigma\, t_1)\ (\sigma\, t_2) \\
    \sigma\, (\lam{x}{t}) & = & \lam{x}{((\sigma \backslash x)\, t)}
  \end{array}
  \]
  \caption{Applying a substitution of closed terms (thus avoiding danger of variable capture).
    $\sigma \backslash x$ is the function that is just like $\sigma$ except that it does not
  map $x$ to anything.}
  
\label{fig:substsigma}
\end{figure}

\begin{corollary}
If $\Gamma\vdash t : T$, then $t \downarrow$.
\end{corollary}
\begin{proof} Let $I(b) = \mathcal{N}$ (Definition~\ref{def:normterms}).  Then by Theorem~\ref{thm:stlcsnd},
  for any $\sigma\in\interp{\Gamma}$, we have $\sigma t \in \interp{T}$.  By Lemma~\ref{lem:norminterp},
  $\interp{T}\subseteq \mathcal{N}$, so $\sigma t\downarrow$.  This implies $t\downarrow$ (proof omitted). % FIXME
  \end{proof}
 
\section{Normalization using a well-founded ordering}

In the previous section, we proved normalization for simply typed
terms as a corollary of soundness of the typing rules with respect to
our realizability semantics.  We can give an interpretation
for base types under which $\interp{T}$ is a set of normalizing
terms, and then soundness (Theorem~\ref{thm:stlcsnd}) does the rest.
This semantic approach is a powerful way to prove normalization for
other type systems, and indeed, once the type systems are powerful
enough, it is essentially the only known way to obtain that result.

In this section, we pursue a different approach, in which we compute a
measure for terms, and then show that we can always reduce a
non-normal term in a way that reduces the corresponding measure.
So we are showing that we can reduce terms in a way that makes this
measure smaller.  And the measure assigned to each term comes from
a well-founded set, so it cannot decrease forever.  Thus, each
term has a finite reduction sequence.


\section{Relational semantics}

Realizability semantics (Section~\ref{sec:stlcrealizability})
interprets types as sets of terms.  We may also interpret types as
relations on terms.  The definition is in Figure~\ref{fig:relsemstlc},
where we assume now that we have $I \in (B \to
\mathcal{P}\ (\textit{Terms} \times \textit{Terms}))$, and we then
define $\interp{-}\in (\textit{Types} \to \mathcal{P}\ (\textit{Terms}
\times \textit{Terms}))$.  The set $\mathcal{P}\ (\textit{Terms} \times
\textit{Terms})$ is the set of all subsets of the cartesian product
$\textit{Terms} \times \textit{Terms}$.  Since such a subset is just a
relation, we are interpreting base types and then types as relations on
terms.

\begin{figure}
  \[
\begin{array}{lll}
   \interp{b} & = & I(b) \\
   \interp{T_1\to T_2} & = & \{ (t_1,t_2)\ |\ \all{(t', t'') \in \interp{T_1}}{(t_1\ t', t_2\ t'') \in\interp{T_2}} \}
\end{array}
  \]
\caption{Relational semantics of types}
\label{fig:relsemstlc}
\end{figure}

\subsection{Examples}

Suppose we have base types $b$ and $b'$, interpreted as just below.
Recall that $t\ \uparrow$ means that $t$ is not normalizing
(Definition~\ref{def:nonnorm}).  The examples will also use some
defined terms from Chapter~\ref{ch:prog}: $\textit{false}$ for
$\lam{x}{\lam{y}{y}}$, \textit{id} for $\lam{x}{x}$, and $\Omega$ for
the diverging term $(\lam{x}{x\ x})\ \lam{x}{x\ x}$.
\[
\begin{array}{lll}
  I(b) & := & \{ (t,t')\ |\ t \betaeq t' \} \\
  I(b') & := & \{ (t,t')\ |\ t \betaeq t' \betaeq \textit{false} \}
\end{array}
\]
\noindent Then we have the following relational facts:
\begin{itemize}
\item $\lam{x}{x\ \Omega}$ and $\lam{x}{x\ \textit{id}}$ are related
  by $\interp{b' \to b}$.  To prove this using the semantics of
  Figure~\ref{fig:relsemstlc}, we must assume we have terms $t$ and
  $t'$ which are related by $\interp{b'}$, and show that
  $(\lam{x}{x\ \Omega})\ t$ is related to $(\lam{x}{x\ \textit{id}})\ t'$ by
  $\interp{b}$.  Since $\interp{b} = I(b)$, the latter may be shown this
  way:
  \[
  (\lam{x}{x\ \Omega})\ t \ \betaeq \  (\lam{x}{x\ \Omega})\ \textit{false} \ \betaeq \ \textit{false}\ \Omega\ \betaeq\ \textit{id} \ \betaeq\ \textit{false}\ \textit{id}\ \betaeq \ (\lam{x}{x\ \textit{id}})\ \textit{false} \ \betaeq \ (\lam{x}{x\ \textit{id}})\ t'
      \]

 \item That same pair of terms is not related by $\interp{b \to b}$,
      which we can show, by the semantics of Figure~\ref{fig:relsemstlc},
      by finding terms $t$ and $t'$ related by $\interp{b}$, but
      where $(\lam{x}{x\ \Omega})\ t$ and $(\lam{x}{x\ \textit{id}})\ t'$
      are not related by $\interp{b}$.  Take $t$ and $t'$ both to be \textit{id},
      and we have:
     \[
     (\lam{x}{x\ \Omega})\ t\ =\ (\lam{x}{x\ \Omega})\ \textit{id}\ \betaeq\ \Omega \ \neq_\beta\ \textit{id}\ \betaeq\ (\lam{x}{x\ \textit{id}})\ \textit{id}\ = \ (\lam{x}{x\ \textit{id}})\ t'
     \]
     \end{itemize}

\section{The Curry-Howard isomorphism}

Curry observed the deep connection between typed lambda calculus and
logic which, developed further by Howard, is known as the Curry-Howard
isomorphism.  The starting point is to connect STLC with minimal
implicational logic.  This logic is for proving formulas of the
following form, where $p$ is from some nonempty set $P$ of atomic
propositions:
\[
F ::= p\ |\ F_1 \to F_2
\]
\noindent This syntax is the same, disregarding the names of the
metavariables, as that for simple types $T$ (introduced at the start
of Section~\ref{sec:synstlc}).  Figure~\ref{fig:minimpl} gives
inference rules for deriving expressions of the form $S \vdash F$,
where $S$ is a list of formulas, taken as assumptions.  These rules
are (again, disregarding differences in the names of the
meta-variables in question) exactly those of STLC, except without
the terms.

\begin{figure}
\[
\begin{array}{lllll}
\infer{S \vdash F}{\textit{Find}\ F\ \textit{in}\ S}

&\ \ &

\infer{S \vdash F_1 \to F_2}{S , F_1 \vdash F_2}

&\ \ &

\infer{S \vdash F_2}{S \vdash F_1\to F_2 \qquad S \vdash F_1}

\\ \\

\infer{\textit{Find}\ F\ \textit{in}\ (S, F)}{\ }

&\ \ &

\infer{\textit{Find}\ F\ \textit{in}\ (S, F')}{\textit{Find}\ F\ \textit{in}\ S}

&\ \ &

\ 

\end{array}
\]
\caption{Proof rules for minimal implicational logic, with rules for looking up an assumption in a list $S$ of formulas}
\label{fig:minimpl}
\end{figure}

Every STLC typing derivation can be translated to a derivation in
minimal implicational logic, assuming that the set $B$ of base types
in STLC is translated to a subset of the set $P$ of atomic
propositions.  For simplicity, in the following example, let us assume
that $B \subseteq P$ (so the translation is the identity function).
Then we may translate the derivation of Figure~\ref{fig:stlcex} to the
proof in Figure~\ref{fig:minimplex}.  The derivation contains an
unnecessary derivation of $S \vdash b$ from the top to about the
middle of the overall derivation.  It is unnecessary because we can
already derive $S \vdash b$ just using the rule for assumptions (first
rule of Figure~\ref{fig:minimpl}).  Does this mean the correspondence
with the STLC derivation is somehow awry?  Not at all.  For we could
just as well derive $\cdot \vdash \lam{x}{\lam{y}{y}} : (b \to b \to
b) \to b \to b$ in STLC.  The structure of the shorter proof in
minimal implicational logic exactly mirrors this simpler lambda term.

Where one may be content to have proved a theorem without minding
too much the details of the proof, in typed lambda calculus the term
that corresponds to a different proof may be a computationally different
function, as in the example just considered: $\lam{x}{\lam{y}{y}}$
behaves very differently, from a computational perspective, from
$\lam{x}{\lam{y}{x\ y\ y}}$.

\begin{figure}
  \[
  \infer{\cdot \vdash  (b \to b \to b) \to b \to b}
        {\infer{\cdot,b \to b \to b \vdash b \to b}
          {\infer{S \vdash b}
            {\infer{S \vdash b \to b}
              {\infer{S \vdash b \to b \to b}{\ }
              & \infer{S \vdash b}{\ }}
            &\infer{S \vdash b}{\ }}}}
  \]
\caption{Example derivation in minimal implicational logic, where $S$ abbreviates $b \to b \to b, b$}
\label{fig:minimplex}
\end{figure}


\section{Exercises}

\subsection{Realizability semantics for types}

\begin{enumerate}
  
\item Suppose $B$ is $\{ b_1, b_2, b_3\}$, and define $I$, recalling
  from Definition~\ref{def:normalizing} that $t\downarrow$ means that $t$
  is normalizing:
\begin{eqnarray*}
I(b_1) & = & \{\ t\ |\ \exists t'.\ t\ \betaa^*\ t'\ \betaa\ t'\ \} \\
I(b_2) & = & \{\ t\ |\ t \downarrow \} \\
I(b_3) & = & \{\ t\ |\ t \betaa^*\ \lam{x}{x} \} \\
\end{eqnarray*}

\noindent Also, define the term $t$ as follows:
\[
t\ =\ \lam{f}{(\lam{x}{x\ x})\ (f\ \lam{x}{x\ x})}
\]

\begin{enumerate}

\item Prove that $t$ is in $\interp{b_2}$.

\item Prove that $t$ is also in $\interp{b_3 \to b_1}$.

\item Prove that $t$ is also in $\interp{(b_2 \to b_3)\to b_3}$.

\item Find a term $t'$ that is in $\interp{(b_3 \to b_2) \to b_2}$
  and also in $\interp{b_1 \to b_1}$; please explain why your term is in both those sets. 

\end{enumerate}
\end{enumerate}
 
\subsection{Type assignment rules}
\label{sec:stlcextp}

\begin{enumerate}

\item Write out typing derivations, using the rules of Figure~\ref{fig:stlctpassign}, for
  the following typing judgments, assuming base types $a$, $b$, and $c$.  You do not need
  to write out the derivations for the \textit{Find} judgment for looking up typings of variables in the
  context.
  \begin{enumerate}
  \item $\cdot, x:b, y : b\to b \vdash y\ (y\ x) : b$
  \item $\cdot \vdash \lam{x}{\lam{y}{x}} : a \to b \to a$
  \item $\cdot \vdash \lam{x}{\lam{y}{\lam{z}{x\ z\ (y\ z)}}} : (a \to b \to c) \to (a \to b) \to a \to c$
  \end{enumerate}
\end{enumerate}  

\subsection{Relational semantics}

\begin{enumerate}
\item Suppose we have a base type $b$, and let $I(b)$ be
  \[
  \{ (t,t')\ |\ (t\ t')\ \downarrow \}
  \]
  \noindent Recall that $t\ \downarrow$ means that $t$ is normalizing (Definition~\ref{def:normalizing}).

  \begin{enumerate}
  \item Argue in detail that $\lam{x}{\lam{y}{x\ (y\ \textit{id})}}$ and $\lam{y}{\lam{z}{z\ y}}$
    are related by $\interp{b \to b}$.

  \item Give another example of a pair of terms in $\interp{b \to b}$.  Please argue in detail for membership in this relation.
  \end{enumerate}
\end{enumerate}

\subsection{Curry-Howard isomorphism}

\begin{enumerate}
\item Translate the typing derivations you did in Section~\ref{sec:stlcextp} above, into proofs in minimal implicational logic.
  \end{enumerate}
