\chapter{Graph Representations of Lambda Terms}

As we have seen so far, scoping, capture avoidance, and syntactic
substitution have been central technical matters to resolve in
defining lambda calculus.  In this chapter, we explore an alternative
which avoids almost all of that complexity, namely graph
representations of lambda terms.  Instead of representing a term as a
tree, we may represent it as a more general kind of graph.  Instead of
having a name, a binding occurrence of a variable will be represented
by an edge in the graph from the $\lambda$ node to the place the
variable is used.  In order to accommodate terms with multiple bound
occurrences for a particular $\lambda$ node, we must introduce
duplicator nodes that explicitly fork such an edge.  We will make
use of the traditional syntax using bound variables in subsequent
chapters, but it is hopefully enlightening to see this alternative
view.

\subsection{Graphical syntax}

A lambda term in graphical syntax has three kinds of nodes, shown in Figure~\ref{fig:lamgraphs}.  These
may look similar to the tree syntax of Figure~\ref{fig:lamtrees}, but the similarity is superficial.
Here, lambda terms are represented by graphs connecting 

\begin{figure}
\begin{center}
\begin{tabular}{lllll}
  \begin{tikzpicture}
    \node at (0,1.414) (a){\ };
    \node at (1.414,-1.414)(b){\ };
    \node at (-1.414,-1.414)(c){\ };
    \node[circle,draw] at (0,0) (n){$\lambda$};
    \path (n) edge (a)
          edge (b)
          edge (c);
\end{tikzpicture}
&\ \ \ \ \ \ \ &
  \begin{tikzpicture}
    \node at (0,1.414) (a){\ };
    \node at (1.414,-1.414)(b){\ };
    \node at (-1.414,-1.414)(c){\ };
    \node[circle,draw] at (0,0) (n){@};
    \path (n) edge (a)
              edge (b)
              edge (c);
\end{tikzpicture}
  &\ \ \ \ \ \ \  &
  \begin{tikzpicture}
    \node at (0,1.414) (a){\ };
    \node at (1.414,-1.414)(b){\ };
    \node at (-1.414,-1.414)(c){\ };
    \node[regular polygon, regular polygon sides=3,draw] at (0,0) (n){\ };
    \path (n) edge (a)
              edge (b)
              edge (c);
\end{tikzpicture}
\end{tabular}
\end{center}
\caption{The three kinds of nodes for graphical syntax of lambda terms}
\label{fig:lamgraphs}
\end{figure}


\section{Exercises}

