\chapter{Introduction}

The formal system known as lambda calculus was invented by Alonzo
Church, and published first in his paper ``A Set of Postulates for the
Foundation of Logic''~\cite{church32}.  As that title suggests,
Church's motivation for devising lambda calculus was to create a
formal foundation for logic and mathematics.  This was in response to
the crisis in foundations of mathematics that occurred in the early
20th century, with the discovery of paradoxes in proposed foundational
theories.  Bertrand Russell's discovery, in 1901, of a contradiction
in the foundational theory being developed by Gottlob Frege was a
prime and motivating example~\cite{Whitehead:268025}.  Church's own
theory was quickly discovered to be inconsistent, as, sadly, was even
a revised version~\cite{church33}.  

One may take these failures as as a cautionary tale of the difficulty
of creating consistent foundational theories.  Or, more inspiringly,
one can understand them as showing that many good results can come
from endeavors that fall short of their objectives.  For from these
early systems of Church, and the work he and his brilliant graduate
students carried out consequently, has arisen a remarkable line of
inquiry, with tremendous theoretical and practical impact, on the
subjects of typed and untyped lambda calculus.  For an engaging
exposition of this history, see the paper by Cardone
and Hindley~\cite{cardone09}.

Church subsequently published a research monograph focused on the
lambda calculus as a formal notion of computation, rather than a
foundation for mathematics~\cite{church41}.  I will take this
monograph to be his definitive presentation of lambda calculus.

\subsection{Why this book}

There are a number of very impressive books on lambda calculus
currently available.  For example, Hendrik Barendregt's book remains
an authoritative source for many deep topics in the theory of untyped
lambda calculus~\cite{barendregt85}, and his more recent book,
co-authored with will Dekkers and Richard Statman, is a similar source
for certain topics in typed lambda calculus~\cite{barendregt+13}.  But
these are reference works, which are far too advanced to serve as
textbooks.  One book on lambda calculus that is at an appropriate
level for university instruction is the one by J. Roger Hindley and
Jonathan Seldin~\cite{hindley+08}.  But this book has a more
mathematical perspective on the subject, and puts less emphasis on
certain more computational points.  So in my opinion, there is
currently no available textbook, for students at the late
undergraduate or early graduate level, on lambda calculus from the
perspective of Computer Science.  And that is what the current
volume seeks to supply.
