\chapter{Programming in Lambda Calculus}
\label{ch:prog}

\section{Basic functions}
\label{sec:basicfuncs}

Let us consider a few basic lambda terms that are useful for
programming in lambda calculus.  We will give names to the terms we
consider, as meta-linguistic abbreviations.  Let us use the syntax $N
:= t$ to indicate (in our meta-language) that we wish to use name $N$
as an abbreviation for term $t$.  In all cases, our choice of names
will be justified by the behavior of the term when applied to various
arguments.  By behavior, I mean the term's $\beta$-reductions.

\paradef{.5}{The identity function.} 
\[
\textit{id} := \lam{x}{x}
\]
\noindent This term really does behave like the mathematical
(set-theoretic, let us say) identity function, since if we apply
\textit{id} to anything, we just get back that same value.\index{identity \textit{id}}  We have

\[
\textit{id}\ t \ \ \betaa^* \ \ t
\]

\paradef{0}{Self-application operator.}
\[
\delta := \lam{x}{x\ x}
\]
\noindent This operator applies input $x$ to $x$. \index{self-application $\delta$}
We also have
\[
\Omega := \delta\ \delta
\]
\noindent This term has no normal form, reducing forever to itself:\index{diverging term $\Omega$}
\[
\begin{array}{ll}
  \underline{\delta}\ \delta & = \\
  \underline{(\lam{x}{x\ x})\ \delta} & \betaa \\
  \delta\ \delta &\
\end{array}
\]

\paradef{0}{Composition.}
\[
\textit{compose} := \lam{f}{\lam{g}{\lam{x}{f\ (g\ x)}}}
\]
\noindent This term can be applied to any terms $f$ and $g$, and it
will return a term that behaves like their composition: it applies $f$
after $g$ to its input $x$.  It is nice to borrow mathematical notation for
function composition and write $f \circ g$ for $\textit{compose}\ f\ g$.
Here are a few examples using \textit{compose}:\index{composition \textit{compose} (or $\circ$)}
\begin{itemize}
\item $\textit{id} \circ \textit{id} \ \ \betaa^*\ \ \textit{id}$
\item $\delta\circ \delta \ \ \betaa^*\ \ \lam{x}{x\ x\ (x\ x)}$:
\[
\begin{array}{ll}
  \underline{\delta\circ \delta} & = \\
  \underline{\textit{compose}}\ \delta\ \delta & = \\
  \underline{(\lam{f}{\lam{g}{\lam{x}{f\ (g\ x)}}})\ \delta}\ \delta & \betaa \\
  \underline{(\lam{g}{\lam{x}{\delta\ (g\ x)}})\ \delta} & \betaa \\
  \lam{x}{\delta\ (\delta\ x)} & \betaa \\
  \lam{x}{\delta\ \underline{(\delta\ x)}} & \betaa \\
  \lam{x}{\underline{\delta\ (x\ x)}} & \betaa \\
  \lam{x}{(x\ x)\ (x\ x)} & \ 
\end{array}
\]
  
\end{itemize}

\paradef{.5}{Application operator.}
\[
\textit{app} := \lam{x}{\lam{y}{x\ y}}
\]
\noindent This term takes in inputs $x$ and $y$ and returns the result of applying $x$ to $y$.  So it is a term which acts like the term construct of application.

\section{Representing numbers with the Church encoding}
\label{sec:churchenc}

For Church's original goal of a foundation for mathematics, it is
paramount that there is some way to represent natural numbers, and the
intuitively computable operations on them, as lambda terms.  Happily,
there are several such \emph{lambda encodings} for representing data
as lambda terms.  Here we see the first, which is Church's own encoding.\index{Church encoding}

Any lambda encoding must represent data as lambda terms implementing
some behavioral interface.  That is, data are defined by what they do,
not what they are.  The idea of the Church encoding more specifically
is to define numbers as their own iteration functions: functions which
take in another function $f$ and starting point $x$, and repeatedly
call $f$ starting with $x$.  Let us write $\rep{n}$ to mean the lambda
term representing $n\in\mathbb{N}$.  Then the Church encoding defines:

\[
\begin{array}{lll}
  \rep{n} & = & \lam{f}{\lam{x}{\underbrace{f \ (\cdots\  (f}_{n}\ x))}}
\end{array}
\]
\noindent So we have these concrete examples:

\[
\begin{array}{lll}
  0 & := & \lam{f}{\lam{x}{x}} \\
  1 & := & \lam{f}{\lam{x}{f\ x}} \\
  2 & := & \lam{f}{\lam{a}{f\ (f\ x)}} \\
  3 & := & \lam{f}{\lam{a}{f\ (f\ (f\ x))}} \\  
  \multicolumn{3}{l}{\cdots}
\end{array}
\]

\noindent In passing, we may observe that $1$ and $\textit{app}$ are $\alpha$-equivalent
terms.  When representing data as lambda terms, such coincidences
sometimes occur.

\section{Operations on Church-encoded natural numbers}

Let us see now how to define some basic operations on Church-encoded
natural numbers.

\paradef{.5}{Successor.} The mathematical successor
operation on $\mathbb{N}$ takes in $n$
and returns $n+1$ (i.e., the next number).  Here is the definition for
Church-encoded naturals:
\[
\textit{succ}\ :=\ \lam{n}{\lam{f}{\lam{x}{f\ (n\ f\ x)}}}
\]
\noindent To understand this, let us first see an example:
\[
\begin{array}{ll}
  \underline{\textit{succ}}\ 2 & = \\
  \underline{(\lam{n}{\lam{f}{\lam{x}{f\ (n\ f\ x)}}})\ 2} & \leadsto \\
  \lam{f}{\lam{x}{f\ (\underline{2}\ f\ x)}} & = \\
  \lam{f}{\lam{x}{f\ (\underline{\lam{f}{(\lam{x}{f\ (f\ x)})}\ f}\ x)}} & \leadsto \\
  \lam{f}{\lam{x}{f\ (\underline{(\lam{x}{f\ (f\ x)})\ x})}} & \leadsto \\
  \lam{f}{\lam{x}{f\ (f\ (f\ x))}} & = \\
  3
\end{array}
\]
More generally, if $n\in\mathbb{N}$, then $\textit{succ}\ \rep{n}$ reduces to $\rep{n+1}$.

\paradef{.5}{Addition.}  To compute $\rep{m+n}$ from $\rep{m}$ and $\rep{n}$, the idea is similar to that for successor, except
that here we wish to add not just one $f$ to the left of $\underbrace{f \ \cdots\  (f }_{n}\ x)$, but $m$ applications of $f$.  For
then we would have $m+n$ applications of $f$ as desired for $\rep{m+n}$.  And fortunately, $\rep{m}$ itself gives us the power
to apply $f$ $m$ times to some starting value $Q$, by writing $m\ f\ Q$.  Here, we want $n\ f\ x$ for $Q$.  So the definition
of addition is:
\[
\textit{add} \ := \ \lam{m}{\lam{n}{\lam{f}{\lam{x}{m\ f\ (n\ f\ x)}}}}
\]

\paradef{0}{Predecessor.} Kleene was the first to crack the puzzle of how to compute the $\rep{n}$ from $\rep{n+1}$.
His definition is somewhat complicated, so here is a simpler one.  To my knowledge, this is original.  
\[
\begin{array}{lll}
\textit{just} & := & \lam{n}{\lam{j}{\lam{k}{j\ n}}} \\
\textit{pred} & := & \lam{n}{n\ (\lam{m}{\textit{just}\ (m\ \textit{succ}\ 0)})\ 0\ \textit{id}\ 0}
\end{array}
\]
\noindent Writing $F$ for $\lam{m}{\textit{just}\ (m\ \textit{succ}\ 0)}$, let us first see how $2\ F\ 0$ computes:
\[
\begin{array}{ll}
  \underline{2\ F}\ 0 & \leadsto \\
  \underline{(\lam{a}{F\ (F\ a)})\ 0} & \leadsto \\
  F\ (\underline{F\ 0}) & \leadsto \\
  F\ (\textit{just}\ (\underline{0\ \textit{succ}}\ 0)) & \leadsto \\
  F\ (\textit{just}\ (\underline{\lam{a}{a}\ 0})) & \leadsto \\
  F\ (\underline{\textit{just}\ 0}) & \leadsto \\
  \underline{F\ \lam{j}{\lam{k}{j\ 0}}} & \leadsto \\    
  \textit{just}\ (\underline{(\lam{j}{\lam{k}{j\ 0}})\ \textit{succ}}\ 0) & \leadsto \\
  \textit{just}\ (\underline{(\lam{k}{\textit{succ}\ 0})\ 0}) & \leadsto \\
  \underline{\textit{just}\ (\textit{succ}\ 0)} & \leadsto \\
  \lam{j}{\lam{k}{j\ (\textit{succ}\ 0)}} & \ 
\end{array}
\]
\noindent Now here is the reduction for $\textit{pred}\ 2$:
\[
\begin{array}{ll}
  \underline{\textit{pred}\ 2} & \leadsto \\
  \underline{2\ F\ 0}\ \textit{id}\ 0 & \leadsto^* \textnormal{[by above reduction sequence]}\\
  \underline{(\lam{j}{\lam{k}{j\ (\textit{succ}\ 0)}})\ \textit{id}}\ 0 & \leadsto \\
  \underline{(\lam{k}{\textit{id}\ (\textit{succ}\ 0)})\ 0} & \leadsto \\
  \underline{\textit{id}\ (\textit{succ}\ 0)} & \leadsto \\
  \underline{\textit{succ}\ 0} & \leadsto^* \\
  1 &\ 
\end{array}
\]
  

\paradef{0}{Another predecessor.}  Here is a different, trickier definition of predecessor, which one can find online (sadly, I do
not know who invented it):
\[
\textit{pred}\ :=\ \lam{n}{\lam{f}{\lam{x}{n\ (\lam{g}{\lam{h}{h\ (g\ f)}})\ (\lam{h}{x})\ \textit{id}}}}
\]
\noindent To understand how this works, let us write $F$ for $\lam{g}{\lam{h}{h\ (g\ f)}}$, and $A$ for $\lam{h}{x}$, and
see how $3\ F\ A$ computes:
\[
\begin{array}{ll}
  \underline{3\ F}\ A & \leadsto \\
  (\lam{x}{F\ (F\ (F\ x))})\ A & \leadsto \\
  F\ (F\ (\underline{F\ A})) & = \\
  F\ (F\ (\underline{(\lam{g}{\lam{h}{h\ (g\ f)}})\ \lam{h}{x}})) & \leadsto \\  
  F\ (F\ (\lam{h}{h\ \underline{((\lam{h}{x})\ f)}})) & \leadsto \\
  F\ (\underline{F}\ (\lam{h}{h\ x})) & = \\
  F\ (\underline{(\lam{g}{\lam{h}{h\ (g\ f)}})\ (\lam{h}{h\ x})}) & \leadsto \\      
  F\ (\lam{h}{h\ \underline{((\lam{h}{h\ x})\ f)}}) & \leadsto \\
  \underline{F}\ (\lam{h}{h\ (f\ x)}) & = \\
  \underline{(\lam{g}{\lam{h}{h\ (g\ f)}})\ (\lam{h}{h\ (f\ x)})} & \leadsto \\
  \lam{h}{h\ \underline{((\lam{h}{h\ (f\ x)})\ f)}} & \leadsto \\
  \lam{h}{h\ (f\ (f\ x))} & \
\end{array}
\]

What is happening here?  We see that $3\ F\ A$ has reduced to something similar to $f\ (f\ (f\ x))$, but
with a critical twist: we have $\lambda$-abstracted away the function for the first call to $f$, leaving
the other calls intact.  This gives us what we could think of as a ``flexible'' version of $f\ (f\ (f\ x))$, where we get
to choose which function to call instead of $f$ for the outer application.  And the definition of predecessor makes use of this flexibility
by applying the whole result to \textit{id}.  That produces, then, just $f\ (f\ x)$.  So, understanding $F$ and $A$ to be grafted
into the expression on the second line below (capturing their free variables $f$ and $x$), we have
\[
\begin{array}{ll}
  \textit{pred}\ 3 & \leadsto^* \\
  \lam{f}{\lam{x}{\underline{3\ F\ A}\ \textit{id}}} & \leadsto^* \\
  \lam{f}{\lam{x}{\underline{(\lam{h}{h\ (f\ (f\ x))})\ \textit{id}}}} & \leadsto \\
  \lam{f}{\lam{x}{\underline{(\textit{id}\ (f\ (f\ x)))}}} & \leadsto \\
  \lam{f}{\lam{x}{f\ (f\ x)}} & = \\  
  2 &\ 
\end{array}
\]

In some ways this is similar, as perhaps is inevitable, to the first
version of predecessor we saw: a value is computed from $\rep{n}$ that
is like $\rep{n}$ but allows calling another function -- in particular, \textit{id} -- instead of a
final successor.  In this second version of predecessor, that
value is computed underneath bindings of $f$ and $x$, so that
\textit{id} gets called on applications of $f$ to $x$.  In the
first version of predecessor, \textit{id} gets called on the entire
predecessor term, including bindings of $f$ and $x$.

\section{Representing booleans}
\label{sec:bool}

A simpler datatype than that of the natural numbers is the boolean
type, with values \textit{true} and \textit{false}.  The Church
encoding of this type is
\[
\begin{array}{lll}
  \textit{true} & := & \lam{x}{\lam{y}{x}} \\
  \textit{false} & := & \lam{x}{\lam{y}{y}}
\end{array}
\]
\noindent Each boolean accepts two inputs (one at a time), and returns
one of these.  \textit{true} returns the first, while \textit{false}
returns the second.  Based on this idea, it is easy to see how to define
various boolean operations:
\[
\begin{array}{lll}
  \textit{not} & := & \lam{x}{x\ \textit{false}\ \textit{true}} \\
  \textit{and} & := & \lam{x}{\lam{y}{x\ y\ \textit{false}}}
\end{array}
\]
\noindent We negate (with \textit{not}) a boolean by applying it to
\textit{false} and then \textit{true}.  If the boolean itself is
\textit{true}, then it will return the first of these two inputs,
namely \textit{false}; if it is \textit{false}, it will return
\textit{true}.  This is the desired behavior.  Similarly, \textit{and}
takes in inputs \textit{x} and \textit{y}.  It returns the result of
applying \textit{x} to \textit{y} and then \textit{false}.  If
\textit{x} is \textit{true}, then the result will be \textit{y}; and
this is what we would like for conjunction, since if the first boolean
(\textit{x}) is true, the conjunction's value coincides with the value
of the second (\textit{y}): true if \textit{y} is true, and false if
\textit{y} is false.  And if \textit{x} is \textit{false}, then the
second input (out of \textit{y} and \textit{false}) will be chosen; again,
the desired behavior, since this means conjoining \textit{false} with
anything will reduce to \textit{false}.

It is worth emphasizing that applying boolean operations to values
that are not booleans does not result in an error as it might
in some programming languages.  Here, every lambda term has a well-defined
behavior in the form of its $\beta$-reductions.  But the results of
applications violating the intuitive typings we have in mind for
these operations may be somewhat inscrutable.

\section{Ordered pairs}

It is often convenient to program with a representation of ordered pairs $(x,y)$, given representations of $x$ and $y$.
To construct the representation of a pair, we use this function:
\[
\textit{pair} \ :=\ \lam{x}{\lam{y}{\lam{c}{c\ x\ y}}}
\]
\noindent The idea is that given the components $x$ and $y$ of the
pair, we represent (by the \textit{pair} function) the pair itself as
$\lam{c}{c\ x\ y}$.  This definition embodies the idea that a pair of
$x$ and $y$ is something that can make $x$ and $y$ available for
subsequent computation.  This is done in the encoding by applying the
pair to a function which is expecting the components.

For example, we may define \textit{fst} (``first'') and \textit{snd}
(``second''; these names are often used for these operations in
functional programming languages) as follows:
\[
\begin{array}{lll}
  \textit{fst} & := & \lam{p}{p\ \textit{true}} \\
  \textit{snd} & := & \lam{p}{p\ \textit{false}}
\end{array}
\]
\noindent Since \textit{true} returns the first of two arguments, and
\textit{false} the second, they are used to select either the first or
the second component, respectively, when passed as an argument to the
pair.  (As usual, these functions assume input \textit{p} is a pair
of the form $\lam{c}{c\ x\ y}$, and may give unexpected results
if applied to terms not of that form.)

\section{Representing numbers with the Scott encoding}

In Section~\ref{sec:churchenc}, we saw an elegant representation of
numbers as lambda terms, called the Church encoding.  Every number $n$
is represented as the $n$-fold composition operator.  While many
functions are concisely definable this way, the predecessor operation
required quite some ingenuity, and is asymptotically less efficient
than we might reasonably expect (taking time linear in $n$, instead of
constant time).  In this section, we consider an alternative lambda
encoding due to Dana Scott, which has a straightforward constant-time
predecessor.  With the Scott encoding, each number can be thought of
as a function $t$ that informs the caller whether $t$ represents a
successor number or zero.  In the former case, it also provides the
caller with the representation of the predecessor. The definition is:
\[
\begin{array}{lll}
  0 & := & \lam{f}{\lam{x}{x}} \\
  1 & := & \lam{f}{\lam{x}{f\ 0}} \\
  2 & := & \lam{f}{\lam{x}{f\ 1}} \\
  \multicolumn{3}{l}{\cdots} \\
  \rep{n+1} & := & \lam{f}{\lam{x}{f\ \rep{n}}} \\
  \multicolumn{3}{l}{\cdots} 
\end{array}
\]
So every $\rep{n}$ accepts two inputs $f$ and $x$, and if
$n$ is $0$, returns $x$; and if $n$ is $m+1$ for some $m$, returns
$f\ \rep{m}$.  This makes available the predecessor $\rep{m}$, and
thus the actual predecessor function is easily defined:
\[
\textit{pred} \ := \ \lam{n}{n\ \textit{id}\ 0}
\]
\noindent Here, \textit{id} is passed for $f$ and $0$ for $x$.  This means
that for any Scott-encoded successor number, we have the following reduction:
\[
\begin{array}{ll}
  \underline{\textit{pred}}\ \rep{n+1} & = \\
\underline{(\lam{n}{n\ \textit{id}\ 0})\ \rep{n+1}} & \leadsto \\
\underline{\rep{n+1}}\ \textit{id}\ 0 & = \\
\underline{(\lam{f}{\lam{x}{f\ \rep{n}}})\ \textit{id}}\ 0 & \leadsto \\
\underline{(\lam{x}{\textit{id}\ \rep{n}})\ 0} & \leadsto \\
\underline{\textit{id}\ \rep{n}} & \leadsto \\
\rep{n} & \
\end{array}
\]
\noindent And this is the desired result: $\textit{pred}\ \rep{n+1}
\betaa^* \rep{n}$.  Furthermore, we can see that this reduction required
four steps of $\beta$-reduction, independent of the value of $n$.
This is in contrast to the case with the Church encoding, where the
number of steps was proportional to $n$.

It is obvious from the encoding that the successor function \textit{succ} for Scott-encoded numbers should be:
\[
\textit{succ}\ := \ \lam{n}{\lam{f}{\lam{x}{f n}}}
\]

\section{The Y combinator}
\label{sec:y}

While it is very straightforward to define predecessor on
Scott-encoded numbers, other operations pose a problem.  The Church
encoding takes $n$-fold iteration as the representation of $n$, and
hence has no difficulty defining iterative functions.  Not so the
Scott encoding, and indeed, the only natural way to recurse is to
avail ourselves of a term implementing \emph{general recursion} (this
is recursion that may fail to terminate).  (It should be noted that
there is an extremely tricky way to derive iteration for Scott encodings, but
we will not consider this here~\cite{lepigre+19}.)

General recursion in lambda calculus is provided using a term traditionally
denoted $Y$:
\[
Y \ :=\ \lam{f}{(\lam{x}{f\ (x\ x)})\ (\lam{x}{f\ (x\ x)})}
\]
\noindent This term is usually called a \emph{combinator}, which is an
informal notion indicating that a lambda term is of interest primarily
for use as a building block for defining other functions (as opposed,
say, to implementing some particular algorithm valuable in its own
right).\index{combinator}\index{Y combinator} In this sense, some other terms we have
encountered so far, like identity and composition functions
(\textit{compose}, \textit{id} of Section~\ref{sec:basicfuncs}), are
also reasonably considered combinators.

Terminology aside, let us see how the $Y$ combinator works and how we
can use it to define operations on Scott-encoded numbers.  Suppose $t$ is any
lambda term not containing $x$ free.  Then we have:
\[
\begin{array}{ll}
  \underline{Y}\ t & = \\
  \underline{(\lam{f}{(\lam{x}{f\ (x\ x)})\ (\lam{x}{f\ (x\ x)})})\ t} & \leadsto \\
  \underline{(\lam{x}{t\ (x\ x)})\ (\lam{x}{t\ (x\ x)})} & \leadsto \\
  t\ \underline{((\lam{x}{t\ (x\ x)})\ (\lam{x}{t\ (x\ x)}))} & =_\beta \\
  t\ (Y\ t)
\end{array}
\]
\noindent So we see that $Y\ t$ is $\beta$-equivalent to $t\ (Y\ t)$.  This fact is so important that
it is worth highlighting as an equation:
\[
Y\ t\ =_\beta \ t\ (Y\ t)
\]
\noindent Swapping sides will shortly be revealing:
\[
t\ (Y\ t)\ =_\beta \ Y\ t
\]
\noindent This matches the form of a fixed-point equation for $t$.  In mathematics, a fixed point of a function $F$ is an input $X$ such that
\[
F(X) = X
\]
\noindent Here, with application of lambda terms playing the role of function invocation, and $\beta$-equivalence taking the place of equality,
we can write this as:
\[
F\ X =_\beta X
\]
\noindent For term $t$, this becomes
\[
t\ X =_\beta X
\]
\noindent And indeed, the equation we derived above is of this form, with $Y\ t$ for $X$.

Now what is the significance of this?  It shows us that, contrary to
what we usually find in mathematics, in lambda calculus every function
has a fixed point.  How peculiar!  Certainly some mathematical
functions have fixed points.  Take (mathematical) predecessor on
natural numbers, with the assumption that \textit{pred} of $0$ is $0$.
Then $0$ is a fixed point of \textit{pred}.  But consider boolean negation.
There is no boolean $b$ such that $\textit{not}\ b$ equals $b$ (neither possible
value for $b$, namely \textit{true} or \textit{false}, works).  Strangely, though,
in lambda calculus, we have just seen the general equation of $t\ (Y\ t)$ and $Y\ t$.
This means that
\[
\textit{not}\ (Y\ \textit{not}) =_\beta Y\ \textit{not}
\]
\noindent Something unusual is going on, and indeed, as we will see when we
turn to denotational semantics of lambda calculus, interpreting lambda calculus
in set theory requires significant ingenuity.

But to remain at the linguistic level for the moment, let us try to get an intuition
for how every term $t$ can have $Y\ t$ for a fixed point.  Let us write $U$ for $\lam{x}{t\ (x\ x)}$.  We have seen that
\[
\begin{array}{ll}
  \underline{Y\ t} & \leadsto \\
  \underline{U\ U} & \leadsto \\
  t\ (U\ U)&\ 
\end{array}
  \]
\noindent  This reduction sequence may then be continued as long as we wish:
\[
\begin{array}{ll}
  t\ \underline{(U\ U)}&\leadsto \\
  t\ (t\ \underline{(U\ U)})&\leadsto \\  
  t\ (t\ (t\ \underline{(U\ U)}))&\leadsto \\
  \multicolumn{2}{l}{\cdots}
\end{array}
\]
\noindent If we had some notion of infinite lambda term, we might identify the limit of this infinite reduction sequence,
as this infinite right-nested application of $t$:
\[
t\ (t\ (t\ \cdots
\]
\noindent One can indeed develop an infinitary lambda calculus
allowing infinitary terms like this~\cite{kennaway1997}; but this is
beyond the scope of the book.  But with an infinitary term like
this as an informal guiding intuition, we can see how the fixed-point equation
makes sense.  $Y\ t$ denotes (informally) an infinite
right-nested application of $t$.  Applying $t$ one more time to this
does not change the infinite application, as it is still infinite!

Note that $U\ U$ is a lot like $\delta\ \delta$:
\[
\begin{array}{lll}
  U\ U & = & (\lam{x}{t\ (x\ x)})\ \lam{x}{t\ (x\ x)} \\
  \delta\ \delta & = & (\lam{x}{x\ x})\ \lam{x}{x\ x}
\end{array}
\]
\noindent We have just inserted $t$, but otherwise retain the central idea of self-application
for divergence.

How is this esoterically explained construction useful for programming?  Contrast
the situation with iteration using Church-encoded numbers.  There, $\rep{n}$ gives us the
power to repeat a function $n$ times:
\[
\underbrace{t\ \cdots\ (t}_n\ x)
\]
\noindent But what if we need to repeat a function more times than
just $n$ times?  We could imagine somehow increasing how many times
the composition is iterated, to some bigger number $n'$.  But the most
computationally powerful option is to extend the $n$-fold iteration of
$t$ to an infinite iteration of $t$:
\[
t\ (t\ (t\ \cdots
\]
\noindent But this is just what (informally) $Y\ t$ gives us!  So we
are using the power of diverging computation which we get through
self-application, to allow ourselves as many iterations of $t$ as we
could possibly need.  Fundamental results of recursion theory then
imply that we will of necessity need to accept the possibility of
divergence: we have given ourselves the ability to apply $t$ as many
times as we wish, and we cannot rule out the possibility that it gets
applied infinitely many times with no normal form reachable.

But there is still a puzzle.  How can we ever reach any normal form when
$Y\ t$ has an infinite reduction sequence?  The answer is that existence
of a single infinite reduction sequence does not mean all reduction
sequences are infinite.  Indeed, for a very simple example, consider
\[
(\lam{x}{\lam{y}{y}})\ \Omega
\]
\noindent This term has both an infinite reduction sequence, and also infinitely many finite reduction sequences.  For examples of the first and second, in order, consider:
\[
\begin{array}{l}
  (\lam{x}{\lam{y}{y}})\ \underline{\Omega}\ \leadsto\ (\lam{x}{\lam{y}{y}})\ \underline{\Omega}\ \leadsto \ \cdots \\
  \underline{(\lam{x}{\lam{y}{y}})\ \Omega} \ \leadsto \ \lam{y}{y}
\end{array}
\]
\noindent The normalizing reduction sequence (the second one) drops out the non-normalizing $\Omega$ subterm.
Similarly, in our infinitary term
\[
t\ (t\ (t\ \cdots
\]
\noindent it could happen that there is a reduction to a normal form where an application of $t$ ends up dropping
its argument.  We will see an example next.

\section{Recursive operations on Scott-encoded numbers}

Let us define addition on Scott-encoded numbers using the $Y$ combinator.  The idea is that we wish to implement the following
system of recursive equations, using $Y$ to implement the recursion:
\[
\begin{array}{lll}
  \textit{add}\ 0\ m & = & m \\
  \textit{add}\ (\textit{succ}\ p)\ m & = & \textit{succ}\ (\fbox{\textit{add}}\ p\ m)
\end{array}
\]
\noindent Since the Scott-encoding gives us a way to distinguish whether a number is $0$ or a successor number, we can
easily choose whether between these equations based on the first input.  We then need to use $Y$ to implement the framed
recursion on the right-hand side of the second equation.  The definition is, the following, using helper definition \textit{addh} for
easier consideration below:
\[
\begin{array}{lll}
  \textit{addh} & := & \lam{\textit{add}}{\lam{n}{\lam{m}{n\ (\lam{p}{\textit{succ}\ (\textit{add}\ p\ m)})\ m}}} \\
  \textit{add} & := & Y\ \textit{addh}
  \end{array}
\]
\noindent Let us see how this works with an example, writing $U$ for $\lam{x}{\textit{addh}\ (x\ x)}$:
\[
\begin{array}{ll}
  \underline{\textit{add}}\ 2\ 2 & = \\
  \underline{Y\ \textit{addh}}\ 2\ 2 & \leadsto \\
  \underline{U\ U}\ 2\ 2 & \leadsto \\  
  \underline{\textit{addh}}\ (U\ U)\ 2\ 2 & = \\
  \underline{(\lam{\textit{add}}{\lam{n}{\lam{m}{n\ (\lam{p}{\textit{succ}\ (\textit{add}\ p\ m)})\ m}}})\ (U\ U)}\ 2\ 2 & \leadsto \\  
  \underline{(\lam{n}{\lam{m}{n\ (\lam{p}{\textit{succ}\ (U\ U\ p\ m)})\ m}})\ 2\ 2} & \leadsto^2 \\
  \underline{2}\ (\lam{p}{\textit{succ}\ (U\ U\ p\ 2)})\ 2 & = \\
  \underline{(\lam{f}{\lam{x}{f\ 1}})\ (\lam{p}{\textit{succ}\ (U\ U\ p\ 2)})\ 2} & \leadsto^2 \\
  \underline{(\lam{p}{\textit{succ}\ (U\ U\ p\ 2)})\ 1} & \leadsto \\
  \textit{succ}\ \underline{(U\ U\ 1\ 2)}) & \leadsto^* \\
  \textit{succ}\ (\textit{succ}\ \underline{(U\ U\ 0\ 2)}) & \leadsto^* \\
  \textit{succ}\ (\textit{succ}\ 2) & \leadsto^* \\
  4 &\ 
  \end{array}
\]
\noindent We see in detail that $U\ U\ \rep{n+1}\ m$ reduces to
$\textit{succ}\ (U\ U\ \rep{n}\ m)$.  So we peel successors off the
first argument until we reach $0$, and then we return the second
argument (i.e., $m$).

\section{A direct approach to recursion on Scott-encoded numbers}

Instead of using the $Y$ combinator, it is possible to recurse
directly on Scott-encoded numbers.  A Scott-encoded number takes in a
function $f$ to call with the predecessor if the number is non-zero,
and a value $x$ to return if the number is zero.  The key idea,
attributed by Lepigre and Rafalli to Michel Parigot, is to have $f$
and $x$ themselves expect to be called with $f$ and $x$
again~\cite{lepigre:rafalli19}.  This enables in particular $f$ to
recurse.

Here is a definition of addition for the Scott encoding,
using this idea. I have abstracted out $f$ and $x$ for this
case, to help make clear where there is a self-application
happening.  The base case $x$ depends, in the definition of addition,
on the second addend, so we need to write $x\ m$ in the definition
of \textit{add}, instead of just $x$.
\begin{eqnarray*}
f & := & \lam{p}{\lam{s}{\lam{z}{\textit{succ}\ (p\ s\ z\ s\ z)}}} \\
x & := & \ \lam{m}{\lam{s}{\lam{z}{m}}} \\
\textit{add} & := & \lam{n}{\lam{m}{n\ f\ (x\ m)\ f\ (x\ m)}}
\end{eqnarray*}

\noindent Let us see this definition in action:

\[
\begin{array}{ll}
  \underline{\textit{add}\ 2\ 2} & \leadsto^2 \\
  \underline{2\ f}\ (x\ 2)\ f\ (x\ 2) & \leadsto \\
  \underline{f\ 1\ f\ (x\ 2)} & \leadsto^3 \\
  \textit{succ}\ \underline{(1\ f\ (x\ 2)\ f\ (x\ 2))} & \leadsto^4 \\
  \textit{succ}\ (\textit{succ}\ \underline{(0\ f\ (x\ 2)\ f\ (x\ 2))}) & \leadsto \\
  \textit{succ}\ (\textit{succ}\ \underline{((x\ 2)\ f\ (x\ 2))}) & \leadsto^3 \\
  \underline{\textit{succ}\ (\textit{succ}\ 2)} & \leadsto^* \\    
  4
\end{array}
  \]

\section{Exercises}

\subsection{$\beta$-reductions for some simple terms}

\begin{enumerate}
\item For each of the following terms, write down a $\beta$-normal form to which the term reduces.  You do not need to write out all the steps in a $\beta$-reduction sequence.  Please just give a $\beta$-normal form.

  \begin{enumerate}
  \item $\textit{app}\ \circ\ \textit{id}$
\vspace{.5cm}
  \item $\textit{app}\ \circ\ \textit{app}$
\vspace{.5cm}
  \item $\lam{z}{2\ z}$
\vspace{.5cm}
  \item $\delta\ \textit{app}$
\vspace{.5cm}
  \item $2\ 2$
\vspace{.5cm}
  \item $\textit{not}\ 2$ (as noted above, terms like this which violate intuitive typings do have a
    well-defined behavior)
\vspace{.5cm}
  \item $\textit{and}\ 1$
  \end{enumerate}

\item Please write out a maximal $\beta$-reduction sequence (renaming is not necessary, so we can use just $\betar$
  instead of $\betaa$) starting with $\textit{pred}\ 2$.

\end{enumerate}

\subsection{Programming in lambda calculus}

\begin{enumerate}
\item Define a disjunction operator (i.e., boolean ``or'') on Church-encoded booleans, and demonstrate that it is working by writing a maximal $\betaa$-reduction sequence starting with $\textit{or}\ \textit{false}\ \textit{true}$.

\item \ \textbf{[Challenge]} Find an alternative definition of \textit{pred} with similar form as above, namely
  \[
  \lam{n}{\lam{f}{\lam{x}{n\ F'\ A'\ t_1 \cdots t_k}}}
  \]
  \noindent for some terms $F'$ and $A'$ grafted into this expression (which hence might have free occurrences of $f$ or $x$ that get bound by the $\lambda$-abstractions of those two variables), and some extra terms $t_1, \cdots, t_k$.
  The critical requirement is that where $\rep{n+1}\ F\ A$ reduces (with the definition of $F$ and $A$ in the text) to $\lam{h}{h\ (\underbrace{f \cdots (f}_n\ x))}$, your version with your $F'$ and $A'$ and some of your extra terms should
  reduce to $\lam{h}{\underbrace{f \cdots (f}_n\ (h\ x))}$.

\item Define a function \textit{flip} which reverses the order of components in a pair.

\item Define a subtraction operation on Scott-encoded numbers.  Your term is free to invoke \textit{pred} for predecessor,
  and the $Y$ combinator (and other terms we have defined so far, if you wish).

  \item The term $Y\ Y$ has many different (infinite) reductions.  Try
    to indicate a little of the complexity of this term by showing
    prefixes of some of its reduction sequences.  It would be
    interesting to organize these initial parts of reduction sequences
    into a tree, so we can see how reduction can branch out in different
    ways from the starting point of $Y\ Y$.  (This problem is not concerned
    with giving an exact correct answer, but rather with showing that you
    have explored the reduction behavior of this rather exotic term.)
\end{enumerate}
